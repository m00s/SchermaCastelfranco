\subsubsection{Definizione}
	\noindent La parte \underline{utente} \`e composta da 7 pagine:
\begin{itemize}
	\item {\bfseries\textit{Articoli}} che mostra gli articoli recentemente inseriti.
	\item {\bfseries\textit{Articolo}} che mostra un singolo articolo selezionato.
	\item {\bfseries\textit{Documenti}} che rende scaricabili dei documenti di interesse.
	\item {\bfseries\textit{Storia}} che descrive la storia della societ\`a.
	\item {\bfseries\textit{Staff}} che mostra le persone che fanno parte dello staff.
	\item {\bfseries\textit{Corsi}} che mostra l'orario dei corsi.
	\item {\bfseries\textit{Login}} che mostra un form di login per l'accesso all'area privata.
\end{itemize}
	In particolare le pagine \textit{staff.html}, \textit{storia.html} e \textit{corsi.html} sono statiche, mentre le altre pagine sono state ottenute attraverso l'associazione dei file xml ad un foglio di stile e la stampa tramite Perl.

La parte \underline{amministratore} \`e composta da 6 pagine:
\begin{itemize}
	\item {\bfseries\textit{InserisciArticolo.cgi}} che permette di inserire un nuovo articolo.
	\item {\bfseries\textit{ModificaArticolo.cgi}} che permette di scegliere un articolo da modificare.
	\item {\bfseries\textit{EliminaArticolo.cgi}} che permette di selezionare articoli da eliminare.
	\item {\bfseries\textit{InserisciDocumento.cgi}} che permette di inserire un nuovo documento.
	\item {\bfseries\textit{ModificaDocumento.cgi}} che permette di scegliere un documento da modificare.
	\item {\bfseries\textit{EliminaDocumento.cgi}} che permette di selezionare documento da eliminare.
\end{itemize}	
	
	Tutte queste pagine sono dinamiche, create in Perl come composizione di documenti definiti in XHTML e istruzioni Perl.
	
\subsubsection{Database}
	La persistenza dei data \`e ottenuta attraverso due file XML validi secondo gli schemi definiti in \textit{articoli.xsl} e \textit{documenti.xsd}. Si \`e scelto XML Schema poich\`e garantisce l'espressivit\`a di cui abbiamo bisogno, in particolare per avere a disposizione i primitivi date e anyURI, utili nella definizione di articoli e documenti.