\subsection{Link nascosti}
Per rendere il sito pi\`u accessibile e migliorare la navigazione, in particolar modo per le persone non vedenti o comunque per tutti coloro che utilizzano come supporto uno screen reader, abbiamo deciso di introdurre numerosi link nascosti che consentono di muoversi agilmente all'interno di una determinata pagina web. \\
Ecco nel dettaglio la loro implementazione:
	\begin{itemize}
		\item
		\item
		\item
	\end{itemize}
\subsection{Lista nascosta}
Per facilitare la navigazione tra gli articoli e i documenti è stata inserita una lista nascosta di link (prima del contenuto) che rimandano al testo esatto selezionato.

\subsection{Tab Index}
Un altro meccanismo utilizzato per consentire una migliore esperienza per l'utente \`e stato quello di personalizzare i tab index in modo che ci si possa muovere pi\`u velocemente e senza avere bisogno del mouse.
In particolare gli abbiamo utilizzati per la navigazione nel men\`u e per i form presenti nella parte amministrativa.

\subsection{Breadcrumb}
Per far capire all'utente in che sezione del sito si trova, abbiamo pensato di modificare il colore di sfondo nella lista del menu\`per la pagina selezionata. \\  Per chi utilizza screen reader è stato predisposto un campo nascosto che ti dice dove sei.

\subsection{Contrasti dei colori}

\subsection{Mappa del sito}
E\` stato scelto di non inserire una mappa in quanto il numero di pagine presenti nel sito e\` in numero ristretto e difficilmente l'utente potra\`a perdersi durante la navigazione.