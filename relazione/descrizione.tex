Il gruppo ha deciso di realizzare un sito di pubblicazione di articoli e documenti, per la societ\`a sportiva di scherma di Castelfranco Veneto. Il sito sar\`a composto da una parte utente, accessibile da chiunque, e una parte amministratore, accessibile solamente dalla persona incaricata di pubblicare articoli e documenti. 
Per un utente del sito sar\`a dunque possibile consultare gli articoli recenti, scaricare documenti in formato PDF, e accedere a informazioni pubbliche della societ\`a, mentre l'amministratore potr\`a accedere all'area privata e gestire le risorse pubblicate, inserendo, modificando o eliminando articoli e/o documenti.\\

\noindent La parte utente \`e composta da 7 pagine:
\begin{itemize}
	\item {\bfseries\textit{Home.html}} che mostra una immagine di introduzione al sito.
	\item {\bfseries\textit{Articoli.html}} che mostra gli articoli recentemente inseriti.
	\item {\bfseries\textit{Documenti.html}} che rende scaricabili dei documenti di interesse.
	\item {\bfseries\textit{Storia.html}} che descrive la storia della societ\`a.
	\item {\bfseries\textit{Staff.html}} che mostra le persone che fanno parte dello staff.
	\item {\bfseries\textit{Corsi.html}} che mostra l'orario dei corsi.
	\item {\bfseries\textit{Login.html}} che mostra un form di login per l'accesso all'area privata.
\end{itemize}
La parte amministratore \`e composta da 6 pagine:
\begin{itemize}
	\item {\bfseries\textit{InserisciArticolo.cgi}} che permette di inserire un nuovo articolo.
	\item {\bfseries\textit{ModificaArticolo.cgi}} che permette di scegliere un articolo da modificare.
	\item {\bfseries\textit{EliminaArticolo.cgi}} che permette di selezionare articoli da eliminare.
	\item {\bfseries\textit{InserisciDocumento.cgi}} che permette di inserire un nuovo documento.
	\item {\bfseries\textit{ModificaDocumento.cgi}} che permette di scegliere un documento da modificare.
	\item {\bfseries\textit{EliminaDocumento.cgi}} che permette di selezionare documento da eliminare.
\end{itemize}

